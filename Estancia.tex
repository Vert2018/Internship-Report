%% UPY Report template, adapted from bare_jrnl.tex by Michael Shell

%% *************************************************************************
%% Legal Notice:
%% This code is offered as-is without any warranty either expressed or
%% implied; without even the implied warranty of MERCHANTABILITY or
%% FITNESS FOR A PARTICULAR PURPOSE!
%% User assumes all risk.
%% In no event shall the UPY or any contributor to this code be liable for
%% any damages or losses, including, but not limited to, incidental,
%% consequential, or any other damages, resulting from the use or misuse
%% of any information contained here.
%%
%% All comments are the opinions of their respective authors and are not
%% necessarily endorsed by the UPY.
%%
%% This work is distributed under the LaTeX Project Public License (LPPL)
%% ( http://www.latex-project.org/ ) version 1.3, and may be freely used,
%% distributed and modified. A copy of the LPPL, version 1.3, is included
%% in the base LaTeX documentation of all distributions of LaTeX released
%% 2003/12/01 or later.
%% Retain all contribution notices and credits.
%% ** Modified files should be clearly indicated as such, including  **
%% ** renaming them and changing author support contact information. **
%% *************************************************************************


\documentclass[peerreviewca]{IEEEtran}
% Be sure to compile with pdflatex, not standard latex


% *** Required packages, do not change ***
\usepackage{cite}
\usepackage{graphicx}
% Ensure all graphic files are inside the figures directory
\graphicspath{{./figures/}}
\usepackage{amsmath}
\interdisplaylinepenalty=2500
\usepackage{url}
\usepackage[caption=false,font=footnotesize]{subfig}

% IEEEtran contains the IEEEeqnarray family of commands that can be used to
% generate multiline equations as well as matrices, tables, etc., of high
% quality. You can alfo use the \align command of the amsmath package.

% ** SPECIALIZED LIST PACKAGES ***
%
\usepackage{algorithmic}
% algorithmic.sty was written by Peter Williams and Rogerio Brito.
% This package provides an algorithmic environment fo describing algorithms.
% You can use the algorithmic environment in-text or within a figure
% environment to provide for a floating algorithm. Do NOT use the algorithm
% floating environment provided by algorithm.sty (by the same authors) or
% algorithm2e.sty (by Christophe Fiorio) as the IEEE does not use dedicated
% algorithm float types and packages that provide these will not provide
% correct IEEE style captions. The latest version and documentation of
% algorithmic.sty can be obtained at:
% http://www.ctan.org/pkg/algorithms
% Also of interest may be the (relatively newer and more customizable)
% algorithmicx.sty package by Szasz Janos:
% http://www.ctan.org/pkg/algorithmicx



% *** Do not adjust lengths that control margins, column widths, etc. ***
% *** Do not use packages that alter fonts (such as pslatex).         ***

% correct bad hyphenation here
\hyphenation{op-tical net-works semi-conduc-tor}


\begin{document}
%
% paper title
% Titles are generally capitalized except for words such as a, an, and, as,
% at, but, by, for, in, nor, of, on, or, the, to and up, which are usually
% not capitalized unless they are the first or last word of the title.
% Linebreaks \\ can be used within to get better formatting as desired.
% Do not put math or special symbols in the title.

\title{Name of Project}

% author names
% note positions of commas and nonbreaking spaces ( ~ ) LaTeX will not break
% a structure at a ~ so this keeps an author's name from being broken across
% two lines.

\author{\IEEEauthorblockN{Student~Name}
  \IEEEauthorblockA{Data Engineering\\
    Universidad Polit\'ecnica de Yucat\'an\\
    Km. 4.5. Carretera M\'erida --- Tetiz\\
    Tablaje Catastral 4448. CP 97357\\
    Uc\'u, Yucat\'an. M\'exico\\
    Email: student@upy.edu.mx}
  \and
  \IEEEauthorblockN{External~Supervisor~Name}
  \IEEEauthorblockA{Twentieth Century Fox\\
    Springfield, USA\\
    Email: homer@thesimpsons.com}
  \and
  \IEEEauthorblockN{Internal~Supervisor~Name}
  \IEEEauthorblockA{Universidad Polit\'ecnica de Yucat\'an\\
    Km. 4.5. Carretera M\'erida --- Tetiz\\
    Tablaje Catastral 4448. CP 97357\\
    Uc\'u, Yucat\'an. M\'exico\\
    Email: professor@upy.edu.mx}}

% The paper headers
\markboth{Internship Report, Spring 2019, UPY}%
{Student Last Name: Title}
% The only time the second header will appear is for the odd numbered pages
% after the title page when using the twoside option.
%

% Do NOT change this
\IEEEpubid{\today\ \copyright\ Universidad Polit\'ecnica de Yucat\'an}

% make the title area
\maketitle



% As a general rule, do not put math, special symbols or citations in the
% abstract or keywords. The abstract must contain between 200 and 250 words.
\begin{abstract}
  The abstract goes here. 200--250 words. A summary of the whole report
  including important features, results and conclusions.
\end{abstract}

% 
\begin{IEEEkeywords}
  Please provide a few keywords that describe the 
\end{IEEEkeywords}

\vspace*{\fill}
\begin{center}
  \includegraphics[width=3in]{upy-logo}
\end{center}
\vspace{0.5in}
%
% Do not change this
\IEEEpeerreviewmaketitle


\section{Introduction}
% The very first letter is a 2 line initial drop letter followed
% by the rest of the first word in caps.
%
% form to use if the first word consists of a single letter:
% \IEEEPARstart{A}{demo} file is ....


\IEEEPARstart{I}{n} the introduction, you are supposed to highlight the main
aims of the paper to the reader. Let the reader understand the purpose of you
writing the report. You can also comment on the flow of the report so that the
reader can know what to expect.

Justify the importance of the work done during the internship, as well as the
possible impact.

VERY IMPORTANT: MAKE SURE TO INCLUDE AT LEAST 10 bibliographic references, (and
use them through out this document)


% You must have at least 2 lines in the paragraph with the drop letter
% (should never be an issue)

\subsection{Subsection Heading Here}
This is how you include a subsection.


\subsubsection{Subsubsection Heading Here}

Subsubsection text here.


% An example of a floating figure using the graphicx package.
% Note that \label must occur AFTER (or within) \caption.

% \begin{figure}[!t]
%   \centering
%   \includegraphics[width=2.5in]{myfigure}
%   \caption{Simulation results for the network.}
%   \label{fig_sim}
% \end{figure}

% Note that the IEEE typically puts floats only at the top, even when this
% results in a large percentage of a column being occupied by floats.


% An example of a double column floating figure using two subfigures.
% The subfigure \label commands are set within each subfloat command,
% and the \label for the overall figure must come after \caption.
% \hfil is used as a separator to get equal spacing.
% Watch out that the combined width of all the subfigures on a
% line do not exceed the text width or a line break will occur.
%
% \begin{figure*}[!t]
%   \centering
%   \subfloat[Case I]{\includegraphics[width=2.5in]{box}%
%   \label{fig_first_case}}
%   \hfil
%   \subfloat[Case II]{\includegraphics[width=2.5in]{box}%
%   \label{fig_second_case}}
%   \caption{Simulation results for the network.}
%   \label{fig_sim}
% \end{figure*}
%
% Note that often IEEE papers with subfigures do not employ subfigure
% captions (using the optional argument to \subfloat[]), but instead will
% reference/describe all of them (a), (b), etc., within the main caption.
% Be aware that for subfig.sty to generate the (a), (b), etc., subfigure
% labels, the optional argument to \subfloat must be present. If a
% subcaption is not desired, just leave its contents blank,
% e.g., \subfloat[].


% An example of a floating table. Note that, for IEEE style tables, the
% \caption command should come BEFORE the table and, given that table
% captions serve much like titles, are usually capitalized except for words
% such as a, an, and, as, at, but, by, for, in, nor, of, on, or, the, to
% and up, which are usually not capitalized unless they are the first or
% last word of the caption. Table text will default to \footnotesize as
% the IEEE normally uses this smaller font for tables.
% The \label must come after \caption as always.
%
% \begin{table}[!t]
%%   increase table row spacing, adjust to taste
%   \renewcommand{\arraystretch}{1.3}
%   if using array.sty, it might be a good idea to tweak the value of
%   \extrarowheight as needed to properly center the text within the cells
%   \caption{An Example of a Table}
%   \label{table_example}
%   \centering
%%   Some packages, such as MDW tools, offer better commands for making tables
%%   than the plain LaTeX2e tabular which is used here.
%   \begin{tabular}{|c||c|}
%     \hline
%     One & Two\\
%     \hline
%     Three & Four\\
%     \hline
%   \end{tabular}
% \end{table}



\section{Objectives}
Clearly state the original objectives to the internship project, comment whether all of
the objectives were achieved, and, if not, give a brief explanation on which
objectives where not achieved and why.


\section{State of the Art}
Provide a complete, but brief, survey of discipline or disciplines related to your
work. Give a little more detail on the relevant work previously done related to
your own work with adequate references.

\section{Methods and Tools}
Give a detailed description of the methods and tools used for the work
reported. Include software, equipment, experimental procedures, data workflows,
etc.

\section{Development}
Describe the process of the work done during the internship. You can add
sections or subsections accordingly.

\section{Results}
Describe the results of the work done.

\section{Conclusion}
Discuss the results described in the previous section, you may include your own
opinions about why some specific outcome was obtained, how would a different
outcome may be obtained, etc. Make sure to include performance comments about your project.

Write a summary of the main points of the body of the report. Be sure to
highlight the important aspects that the reader may have overlooked. 





% if have a single appendix:
% \appendix[Proof of the Zonklar Equations]
% or
% \appendix  % for no appendix heading
% do not use \section anymore after \appendix, only \section*
% is possibly needed

% use appendices with more than one appendix
% then use \section to start each appendix
% you must declare a \section before using any
% \subsection or using \label (\appendices by itself
% starts a section numbered zero.)
%

\appendices
\section{Proof of the First Zonklar Equation}
Appendix one text goes here.

Appendices are used for information that is relevant for the work, but breaks
the flow of the main body of the report. For example, source code, lengthy
mathematical derivations, table data, etc.

% you can choose not to have a title for an appendix
% if you want by leaving the argument blank
\section{}
Appendix two text goes here.


% use section* for acknowledgment
\section*{Acknowledgment}


The author would like to thank\ldots

% can use a bibliography generated by BibTeX as a .bbl file
% BibTeX documentation can be easily obtained at:
% http://mirror.ctan.org/biblio/bibtex/contrib/doc/
% The IEEEtran BibTeX style support page is at:
% http://www.michaelshell.org/tex/ieeetran/bibtex/
% \bibliographystyle{IEEEtran}
% argument is your BibTeX string definitions and bibliography database(s)
% \bibliography{IEEEabrv,../bib/paper}
%
% <OR> manually copy in the resultant .bbl file
% set second argument of \begin to the number of references
%   (used to reserve space for the reference number labels box)
\begin{thebibliography}{1}

\bibitem{IEEEhowto:kopka}
  H.~Kopka and P.~W. Daly, \emph{A Guide to \LaTeX}, 3rd~ed.\hskip 1em plus
  0.5em minus 0.4em\relax Harlow, England: Addison-Wesley, 1999.

\end{thebibliography}
\end{document}



%%% Local Variables:
%%% mode: latex
%%% TeX-master: t
%%% End:
